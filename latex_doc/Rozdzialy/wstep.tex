\chapter*{Wstęp}
\addcontentsline{toc}{chapter}{Wstęp}
Zastosowanie konwolucyjnej sieci neuronowej do przetwarzania obrazu z kamer zamontowanych wokół 
samochodu autonomicznego to skomplikowane zagadnienie, które stało się głównym tematem rozpatrywanym w niniejszej pracy. Aby uczynić go bardziej zrozumiałym dla czytelnika, autor włożył wiele wysiłku w celu zaprezentowania najważniejszych pojęć teoretycznych, związanych z tematem pracy.

\section*{Cel i zakres pracy}
W ramach pracy stworzono aplikację, której zadaniem jest trening konwolucyjnych sieci neuronowych na podstawie obserwacji wizualnych z wirtualnego toru wyścigowego. Sieć neuronowa otrzymuje obraz z kamery umieszczonej wewnątrz samochodu, a wartości generowane przez sieć są wykorzystywane do sterowania pojazdem. \\
Zakresem pracy są objęte:
\begin{enumerate*}
\item Przedstawienie podstawowych pojęć związanych z samochodami autonomicznymi. Przegląd literatury pod kątem stosowanych typów czujników oraz najważniejszych projektów komercyjnych.
\item Opis teorii z zakresu konwolucyjnych sieci neuronowych.
\item Projekt systemu przeznaczonego do treningu konwolucyjnych sieci neuronowych. Opis narzędzi wykorzystanych przy implementacji.
\item Implementacja systemu przeznaczonego do treningu konwolucyjnych sieci neuronowych na podstawie obrazu z kamery umieszczonej wewnątrz wirtualnego samochodu.
\item Wykonanie eksperymentów obliczeniowych na opracowanym systemie oraz analiza uzyskanych wyników.
\end{enumerate*}

\newpage
\section*{Struktura pracy}
Struktura pracy została ukształtowana według tej samej konwencji, jaka została przyjęta dla poprzedniej pracy autora \cite{galios:thesis}. Cytując: ,,\textit{Praca składa się z pięciu numerowanych rozdziałów. Każdy rozdział dotyczy konkretnego aspektu omawianego tematu, a kolejność rozdziałów realizuje zasadę od ogółu do szczegółu. Praca rozpoczyna się od rozdziałów omawiających ogólne zagadnienia teoretyczne,
niezbędne do zrozumienia dalszych rozdziałów pracy. Kończy się natomiast rozdziałem będącym opisem bardzo konkretnych i praktycznych aspektów tematu.}''

Lista rozdziałów przedstawia się następująco:
\begin{enumerate*}
\item \textbf{Samochody autonomiczne} \\
Opis problematyki zagadnienia samochodów autonomicznych, ze szczególnym naciskiem na omówienie urządzeń wykorzystywanych w modelu percepcji samochodów. Pokaźna część rozdziału została poświęcona na opisanie wiodących rozwiązań oferowanych na rynku.
\item \textbf{Konwolucyjne sieci neuronowe} \\
Przegląd subiektywnie wybranych pojęć teoretycznych, które zostały uznane przez autora pracy za istotne w kontekście zrozumienia konwolucyjnych sieci neuronowych oraz ich wykorzystania w dalszej części pracy.
\item \textbf{Projekt systemu i opis narzędzi} \\
Rozdział poświęcony opisowi architektury systemu stworzonego w ramach pisania tej pracy. Dodatkowo, w rozdziale znalazł się opis narzędzi użytych do implementacji systemu.
\item \textbf{Opis implementacji} \\
Rozdział zawiera szczegóły dotyczące implementacji systemu stworzonego w ramach pisania pracy. Szczególny nacisk został położony na opisanie kluczowych modułów systemu.
\item \textbf{Eksperymenty obliczeniowe} \\
Tutaj znajdują się szczegóły związane z eksperymentami obliczeniowymi, przeprowadzonymi w ramach tworzenia niniejszej pracy. W rozdziale znalazł się m.in. opis metodyki przyjętej dla eksperymentów obliczeniowych, opis przebiegu eksperymentów oraz wnioski z uzyskanych wyników.
\end{enumerate*}
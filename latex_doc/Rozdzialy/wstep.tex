\chapter*{Wstęp}
\addcontentsline{toc}{chapter}{Wstęp}
Przedmiotem pracy jest zastosowanie konwolucyjnej sieci neuronowej oraz technik głębokiego uczenia maszynowego do rozwiązania problemu sterowania samochodem autonomicznym. Ponieważ jest to bardzo złożone zagadnienie, dlatego wiele uwagi zostało poświęconej na potrzeby dobrego zrozumienia jego podstawowych elementów.

\vspace*{1.5cm}
\section*{Cel i zakres pracy}
Jako cel pracy przyjęto opracowanie prostego systemu uczącego konwolucyjne sieci neuronowe w oparciu o symulacje uruchamiane w przygotowanym środowisku wirtualnym. Sieć neuronowa rozwiązuje problem jazdy samochodem po torze wyścigowym w oparciu o obraz z kamer zamontowanych w samochodzie. \\
Zakres pracy obejmuje poniższe zagadnienia:
\begin{enumerate*}
\item Przegląd literatury na temat samochodów autonomicznych, wykorzystywanych typów czujników oraz wiodących rozwiązań na rynku.
\item Przegląd literatury na temat konwolucyjnych sieci neuronowych.
\item Zaprojektowanie systemu uczącego konwolucyjne sieci neuronowe oraz wybór odpowiednich narzędzi do jego implementacji.
\item Implementacja systemu uczącego konwolucyjne sieci neuronowe w oparciu o obraz z kamer zamontowanych w wirtualnym samochodzie oraz inne sygnały płynące ze środowiska symulacji.
\item Przeprowadzenie eksperymentów obliczeniowych na utworzonym systemie oraz analiza uzyskanych wyników.
\end{enumerate*}

\newpage
\section*{Struktura pracy}
Praca jest złożona z pięciu numerowanych rozdziałów, a każdy z nich dotyczy określonego elementu omawianego tematu. Kolejność rozdziałów jest realizacją zasady ,,\textit{od ogółu do szczegółu}'' - praca rozpoczyna się od omówienia teoretycznych pojęć, niezbędnych do zrozumienia jej dalszej treści. Natomiast dwa ostatnie rozdziały dotyczą bardzo konkretnych oraz praktycznych zagadnień. Lista rozdziałów przedstawia się następująco:
\begin{enumerate*}
\item \textbf{Samochody autonomiczne} \\
Rozdział opisuje problematykę zagadnienia samochodów autonomicznych, kładąc szczególny nacisk na zaprezentowanie oraz porównanie ze sobą urządzeń wykorzystywanych w modelu percepcji samochodów. Pokaźna część rozdziału została poświęcona opisaniu wiodących rozwiązań oferowanych obecnie na rynku.
\item \textbf{Konwolucyjne sieci neuronowe} \\
Rozdział zawiera najistotniejsze zagadnienia teoretyczne z zakresu konwolucyjnych sieci neuronowych, ich treningu oraz obszaru zastosowań. Przedstawione zagadnienia to niezbędne minimum, potrzebne do prawidłowego zrozumienia dalszej części pracy.
\item \textbf{Projekt systemu i opis narzędzi} \\
Rozdział został podzielony na dwie części. W pierwszej z nich zostały opisane główne założenia projektowe dla implementowanego systemu. Z kolei w części drugiej zamieszczono opis narzędzi wykorzystanych podczas implementacji systemu.
\item \textbf{Opis implementacji} \\
Rozdział przedstawia opis implementacji systemu wykonanego na potrzeby tej pracy. Szczególny nacisk został położony na opisanie kluczowych komponentów wchodzących w skład systemu.
\item \textbf{Eksperymenty obliczeniowe} \\
Rozdział poświęcony na przedstawienie metodyki przeprowadzanych eksperymentów obliczeniowych oraz analizę uzyskanych wyników.
\end{enumerate*}
\chapter*{Podsumowanie}
\addcontentsline{toc}{chapter}{Podsumowanie}
Celem pracy było stworzenie prostego systemu uczącego konwolucyjne sieci neuronowe na bazie symulacji uruchamianych w zaprojektowanym środowisku. Zadaniem wytrenowanych modeli była jazda wirtualnym modelem samochodu po przygotowanym torze wyścigowym. Trening sieci neuronowych odbywał się przy użyciu algorytmu PPO, wykorzystującego technikę uczenia maszynowego o nazwie \textbf{Uczenie ze Wzmocnieniem} (z ang. \textit{Reinforcement Learning}).

Cel pracy został osiągnięty w całości. Stworzona aplikacja pozwala na wytrenowanie sieci neuronowej na wybranym Środowisku Uczenia. Sieć konwolucyjna została z sukcesem wykorzystana do przetwarzania obserwacji wizualnych, pochodzących z kamery zamontowanej na fotelu kierowcy. Wytrenowane sieci neuronowe potrafią pokonywać wirtualne tory wyścigowe - nawet te o dużym stopniu skomplikowania. Więcej informacji o wytrenowanych modelach należy szukać w rozdziale \ref{ExperimentsChapter}-tym.

\section*{Perspektywy dalszych badań w dziedzinie}
Czas i wysiłek włożony w napisanie niniejszej pracy pozwolił na zaledwie pobieżne omówienie podjętego tematu. Uczenie maszynowe oraz projektowanie samochodów autonomicznych to dwa bardzo obszerne zagadnienia, na które wielu wybitnych naukowców poświęciło długie lata badań.

Jednym z oczywistych kierunków dalszych badań byłoby zastosowanie zdobytej wiedzy do wytrenowania modelu sterującego samochodem poruszającym się w prawdziwym świecie. Dla ułatwienia badań, eksperymenty należałoby rozpocząć od pojazdu o mniejszych gabarytach, czyli wykonanego w pewnej skali. Kolejny pomysł to dodanie większej liczby kamer, zamontowanych dookoła samochodu. Takie rozwiązanie doprowadziłoby do zwiększenia pola widzenia i pewniejszego zachowania podczas precyzyjnych manewrów, np. w ciasnych nawrotach lub podczas parkowania.